\documentclass{article}
\usepackage{graphicx}
\usepackage{geometry}
\usepackage[hidelinks]{hyperref}
% \usepackage{biblatex}

\geometry{
    a4paper,
    total={170mm,257mm},
    left=20mm,
    top=20mm,
}

% \bibliography{references}

\title{Title here}
\author{
  Benjam\'in Gonz\'alez\\
  \small{\href{mailto:b.gonzaleztoledo@tudelft.nl}{B.GonzalezToledo@tudelft.nl}}
  }
\date{\today}

\begin{document}

\maketitle

\section{Introduction}

This document aims to guide you to understand the proposed categorisation of single-actor bicycle crashes and categorise the sample sent to you.

\subsection{What is a Single-actor bicycle crash?}

\section{Bicycle dynamics-oriented classification}

The proposed workflow to classify the crash according to the observed characteristics is composed by three sub-categories: motion, interaction and mechanism (see Figure \ref{fig: mindmap}).

\subsection{Motion of the bicycle}

This is related to the main motion of the rear frame of the bicycle while the crash is occuring.
%
Due to the dynamics of the bicycle, and following the common simplification to its analysis, we find two main motions related to the degrees of freedom: pitch-over and roll-over.
%
Additionally, the roll-over motion includes its own sub-classification according to the side of the motion with respect to the original motion.
% Rephrase this.

\begin{itemize}
    \item \textbf{Pitch-over (P):} The main characteristic of this motion is the rear wheel lifting from the ground and following a trajectory that finishes with the front wheel behind the rear wheel.
    \item \textbf{High-side (H):} Characterised for a sudden deceleration of the wheel while in lateral motion, which leads to a violent roll motion in the opposite direction of the initial motion.
    \item \textbf{Low-side (L):} The human-bicycle system follows an excessive roll motion in the same direction as at the beginning.
\end{itemize}

\section{Interaction with the environment}

This category refers to the forces excerted on the human-bicycle system, where we find external perturbation or no external perturbation.

\begin{itemize}
    \item \textbf{External perturbation (1):} This makes reference to any force excerted on the bicycle different to the required ones to make the system work. 
    \item \textbf{No external perturbation (0):} This refers to internal forces in the human-bicycle system, or changes in the required forces to make the system work (e.g. friction between tyres and road surface).
\end{itemize}

\section{Mechanism of crash}

The mechanisms of crashes are closely related to the interaction with the environment.
%
For this reason, we include two sub-categories for external perturbations and three for non-external perturbations.

External perturbations
\begin{itemize}
    \item \textbf{Impulse (I):} Any excerted perturbation with a duration equal or under 0.5 [ms].
        %
        This is because the average response time for a propioceptive {} is {} [ms].
        %
        Therefore, the human is not capable of perform a controlled action to maintain the balance.
    \item \textbf{Rstrictive (E):} Excerted perturbations that are cyclical of continuous in time, where the human is capable of perferming a response but this is unsucessful.
        %
        This type of perturbation generate the loss of one of the degrees of freedom of the system, usually making it uncontrollable.
\end{itemize}

Non-external perturbations
\begin{itemize}
    \item \textbf{Slide (S):} Directly related to the tyre-ground interaction, makes reference to the event where friction is less than required and the wheel exhibits a motion mostly perpendicular to its orientation.
    \item \textbf{Load transfer (T):} Changes in the system occured by load transfer.
        %
        Usually, this destabilises the system in the longitudinal axis.
    \item \textbf{No control (N):} This makes reference to the situaitons where the rider is unable to maintain the balance in scenarios that are within normal riding conditions.
\end{itemize}

Finally, together with the mechanisms, for external perturbations we can identify the direction where they act: longitudinal or lateral.
%
In line manner, these perturbations and slide mechanism can be differentiated according to the place of occurence within the system (front or rear).


\section{Example}

Use the following video \url{https://youtube.com/shorts/VZibdrdhdgM?feature=shared}.

First, it is observed that the main motion of the bicycle in the crash is related to roll angle.
%
Additionally, it is in the same direction of the beginning of the motion.
%
Therefore, this corresponds to the category low-side (L).


Second, there are no visible perturbations that fit into the categories.
%
For this reason, we categorise it as non-externally perturbed (0).


Third, the mechanisms that best fits into the scenario is slide (S).
%
After a detailed examination, it is reasonable to assume that the event started from the front wheel slide (F).


Finally, the classification of this crash would be \textbf{L0-SF}, front slide low side.


\end{document}
